\documentclass[12pt]{report}
\usepackage{amssymb}
\usepackage{amsmath}
\usepackage[margin=1in]{geometry}
\usepackage{enumitem} 
\usepackage{graphicx}
\usepackage{lingmacros}
\usepackage{tree-dvips}
\usepackage{amsfonts}
\usepackage{xcolor}
\usepackage{mathrsfs}
\usepackage{indentfirst}
\usepackage{CJKutf8}
\usepackage{amsmath}
\pagestyle{plain}

\begin{document}
\begin{CJK}{UTF8}{bsmi}
\title{Preliminary Study for Robot-Assisted Endodontic Treatment}
(Blue lines are contents)
ver2.0
%----------------------------------------------------------------------------------Chapter 1 --------------------------------------------------------------------------
\chapter{Introduction}
(5-6 pages, at most 10 pages)
\section{Motivation}
(Introduction of the endodontic treatment)
\section{The Prospect of this Project and Challenges}
(Move to the infected teeth -> Root canal searching -> Repetitive drilling -> Apex Detection)\\
(Challenges: root canal is small, breakage of files)\\
\section{Problem Definition}
(Previous Work: NCTU, YOMI and even other dental robots)\\
(Focus on cleaning procedure)\\
(Two problem definition: prevent breakage of file, clean thoroughly)\\
\section{Main Contributions of the Thesis}
(Robot-Assisted System Design)\\
(Precaution against Endodontic File Fracture)\\
(Prediction of Direction of Root Canal and Automatic Navigation)\\
\section{Organization of the Thesis}
%----------------------------------------------------------------------------------Chapter 2 --------------------------------------------------------------------------
\chapter{The Root Canal Treatment}
(Procedure)
(Detailed information on paper survey)
%----------------------------------------------------------------------------------Chapter 3--------------------------------------------------------------------------
\chapter{Robot-Assisted System}
\section{Requirement and Specification}
(Payload, resolution and workspace)
\section{System Design – The DentiBot}
(DOF discussion, Robot Arm - Meca500, F/T sensor - Mini40, Customized Handpiece)\\
(Why not RCM machnism?)\\
(Why Meca500 and Mini40?)\\
%----------------------------------------------------------------------------------Chapter 4--------------------------------------------------------------------------
\chapter{Kinematics and Admittance Control}
(No numbers, only variables)\\
\section{Kinematic Analysis}
\subsection{Coordinate Definition}
(0~6, Sensor frame, and tool frame)\\
\subsection{Forward and Inverse Kinematics}
\subsection{Jacobian matrix}
(How to obtain Jacobian matrix in frame 6 by Jacobian matrix in frame 0)\\
\subsection{Tool Center Point}
(How to find RCM)\\
\section{Admittance Control}
\subsection{Gravity Compensation of F/T sensor}
\subsection{Admittance Control based on F/T sensor}
\subsubsection{Control Scheme}
(Block diagram, robot command choice)\\
\subsubsection{Discussion about Affection of Parameter Setting}
(K, Bi, Mi)\\
\subsection{Reference Frame Changing of F/T sensor}  
(From sensor frame to tool frame)\\
%----------------------------------------------------------------------------------Chapter 5--------------------------------------------------------------------------
\chapter{Prediction of direction of Root canal and Automatic Navigation Based on Force and Torque Feedback}
\section{Problem Definition}
(Main cause of surgical failure)\\
\section{The Proposed Method}
(Peg in hole method based on F/T feedback)\\
\section{The Implementation of the method}
(Admittance control + Transformation from robot to tool + Transformation from sensor to tool + Motion Planning: based on admittance control)\\
\section{Parameters Setting}
(Modes: Doctor Dragging and Auto navigation)\\
%----------------------------------------------------------------------------------Chapter 6--------------------------------------------------------------------------
\chapter{Precaution of Endodontic Files Fracture Based on Current Feedback}
\section{Problem Definition}   
(Main cause of Files Fracture, File analysis)\\
\section{The Proposed Method and Theorem}
(CACS2020)\\
(Motion Planning: sections)\\
(Current threshold setting)\\
%----------------------------------------------------------------------------------Chapter 7--------------------------------------------------------------------------
\chapter{Preliminary Experiment Result}
\chapter{Experimental Setup}
(Acrylic root canal model)
\section{Admittance Control}
\section{Automatically Direction Changing}
\section{Repetitive Experiment}
%----------------------------------------------------------------------------------Chapter 8--------------------------------------------------------------------------
\chapter{Conclusions and Future works}
(Patient move tracking, root canals searching)\\
\end{CJK}
\end{document}

