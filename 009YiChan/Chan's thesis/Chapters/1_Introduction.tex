\chapter{Introduction}
In this chapter, the procedure of the endodontic treatment, so called the root canal treatment, is introduced.

The root canal cleaning is a big challenge in and of itself
\section{Motivation}
(Introduce the procedure of the endodontic treatment- Open$\longrightarrow $Clean$\longrightarrow $Fill)
\section{Previous Work and Problem Definition}
(Briefly mention some dental robots)																\\
(Focus on cleaning procedure)																		\\
(Two problem definition: prevent breakage of file, clean thoroughly)								
\section{The Proposed Method}
(Move to the infected teeth$\longrightarrow $Root canal searching$\longrightarrow $Repetitive drilling$\longrightarrow $Apex Detection)		\\
(Challenges: root canal is small, risk of file breakage)
(Solutions: 1. Build a robot;2. Force-guided alignment; 3. Control the file rotation speed)										
\section{Main Contributions of the Thesis}
\begin{enumerate}
	\item	Integrate a 6-DoF robotic manipulator with 6-DoF F/T sensor for performing endodontic treatment.
	\item	Develop a framework for robot alignment regarding the position and orientation of root canal. 
	\item	Protect the endodontic file from fracturing by controlling file rotation speed.
\end{enumerate}
Endodontic therapy, also known as root canal treatment, is performed to prevent a tooth from being infected. According to American association of endodontists, more than 15 million root canal treatments are performed every year [1] . Although this therapy had been so prevalent, the outcome largely depends on the clinician’s experience and expertise. Instruments fracture and perforation are two problems that commonly occur during the therapy. Removal of broken files is both technically difficult and therefore it is important to reduce the probability of fracture [2] . In addition to these problems, root canal treatment also requires repeatedly drilling in order to clean the canal thoroughly (Fig. 1, “Drilling Root Canal” step).  This repetitive action of root canal treatment is tedious and time-consuming. Therefore, we designed an automatic endodontic robot to improve the time-efficiency and to reduce the occurrence of instrument fracture in endodontic surgery.
There is one robotic system that is designed to perform endodontic therapy. In Intelligent Micro Robot Development for Minimum Invasive Endodontic Treatment [3] , they proposed a micro robot performing root canal treatment with the assistance of 3D computer model system. It is designed to accomplish endodontic therapy with path planned according to the 3D model. However, the problem of instruments fracture still remains. 
In this paper, a torque monitoring method is proposed. The main causes of fractured files are torsional fracture and flexural fatigue, account for 55.7\% and 44.3\% separately [5]. Therefore, we use current feedback to keep track of the torque which the file is bearing during the endodontic treatment. This torque monitoring system is implemented on an endodontic robot prototype we built. We primarily focus on the cleaning and shaping step since it’s the key step to a successful root canal treatment. With the robot prototype and torque monitoring system, the possibility of instrument fracture can be reduced. Besides, the repetitive action during the drilling step can be performed by the robot.

\section{Organization of the Thesis}