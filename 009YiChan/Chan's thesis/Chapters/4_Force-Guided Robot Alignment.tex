\chapter{Force-Guided Robot Alignment}
\hspace*{6mm}This chapter follows from the previous issue about integration. We continue to demonstrate some technical solutions for system integration. On top of that, the Forces and Torques (F/T) sensor will be included to discuss. Therefore, you can see the chapter as an operating manual when you simultaneously own a robot arm, an F/T sensor, and an end effector. First of all, we will explain why we need to use the F/T sensor in our project in section \ref{sec:pro def}. Furthermore, we will introduce how to compensate the gravity affection while moving the robot in senction \ref{sec:grav compen}. Admittance control based on F/T sensor will be described in section \ref{sec:adm ctrl}. Coordinate transformation of F/T sensor will be interpreted in section \ref{sec:rfc}. Last but not least, we will discuss affection of setting admittance control parameters in section \ref{sec:affection}.
\section{Problem Definition}
\label{sec:pro def}
\hspace*{6mm}In modern robot-assisted surgery, to simulate the eyes of doctors,  image processing is often considered. With the image's information, the surgeon can obtain real-time data that humans can not observe clearly. Surgeons use the image process to do many things, such as tracking whether the patient is moving, positioning an incision point, and navigating an insertion path. As for root canal treatment, the endodontist checks the number of root canals with computed tomography.  Before starting to clean the pulp, the endodontist tries to insert root files into root canals. Then the endodontist takes the patient to have computed tomography(CT). CT scan can clearly show the tooth with the root files. With the preoperative image, the endodontist can check whether all root canals are all found and determine all root canals' lengths. However, the above image application is preoperative. The endodontist can only use the preoperative image to guess the direction of the insertion. The intraoperative image processing is absent due to the dimension restriction of a tooth. It is difficult to get a real-time image when the dentist inserts a root file into the opened tooth because it will obstacle the observation. Besides, the diameter of a root canal is $1$ mm and is smaller than a root file. It is impossible to get the image from any shooting angle.
\par
To solve the first problem - without image processing, we propose a method based on force-guided. We surveyed the peg-in-hole based on F/T sensor feedback \cite{7743375} because the cleaning procedure is similar to it. On the other hand, we provide two operating modes - "Dragging Mode" and "Self-Alignment Mode" to cooperate with a dentist. In "Dragging Mode", we can hold the handpiece to the desired position and orientation. We hope to use this mode to let the dentist move the handpiece to the injured tooth. In "Self-Alignment Mode", the handpiece can automatically align the root canal direction even though there are many complicated conditions with different teeth.

\section{Integration of F/T sensor}
\label{sec:grav compen}
\begin{figure}[htbp]
\begin{center}
\includegraphics[width=1\linewidth]{Images/gravity compensation.png}
\caption{
Data Analysis of F/T sensor.
}\label{fig:gravity compensation}
\end{center}
\end{figure}
\hspace*{6mm}Gravity compensation is a critical technical issue when combining an F/T sensor with a robot arm and an end effector \cite{8997006}. 
Fundamentally, we should receive stable data when a static F/T sensor bears the same load or force. Nevertheless, our F/T sensor is installed on the robot arm and move with the pose of the robot arm. On account of its mobility, the gravity of the end effector will significantly affect the actual value we received. Moreover, starting without resetting the F/T sensor to zero would lead to an initial value. If we could not analyze the actual value to an initial value and a gravity load, we would obtain an unstable actual value, not to mention the external force caused by contact force.
\par
Therefore, we illustrate a method, which is to analyze the actual value of the F/T sensor to an initial value and gravity load in real-time. It's worth noting that with this approach, we can get the installation angle between the F/T sensor and the robot arm, including assembly error, and we no longer need to reset the F/T sensor to zero every time.
\subsubsection{The Centroid Position of End Effector}
To start with the first equation in Figure \ref{fig:gravity compensation},
\begin{equation}\label{eq:gc_iga}
\begin{split}
\text{Initial value}	+ \text{Gravity Load} 		= \text{Actual Value} \\
\Rightarrow \left\{\begin{matrix}
\boldsymbol{f_0}		+\boldsymbol{f_g}			&= \boldsymbol{f}\\ 
\boldsymbol{\tau_0}	+\boldsymbol{\tau_g}		&= \boldsymbol{\tau}	
\end{matrix}\right.\Rightarrow \left\{\begin{matrix}
f_{0x} 				+f_{gx} 			&= f_x\\ 
f_{0y}				+f_{gz} 			&= f_y\\ 
f_{0z}				+f_{gy} 			&= f_z\\ 
\tau_{0x}			+\tau_{gx} 			&=\tau _x\\ 
\tau_{0y}			+\tau_{gy} 			&=\tau _y \\ 
\tau_{0z}			+\tau_{gz} 			&=\tau _z 
\end{matrix}\right.	\\
\end{split}
\end{equation}
where $\boldsymbol{f}$ and $\boldsymbol{\tau}$ are force and torque vector respectively.
\par\noindent
And, by terms of moment arm formula,
\begin{equation}
\begin{split}
\because
\boldsymbol{\tau_g}	&= \boldsymbol{r} \times \boldsymbol{f_g} \\
\end{split}
\end{equation}
\par
where $\boldsymbol{r}$ denotes the centroid position of the end effector in the sensor frame.
\begin{equation}
\begin{split}
\therefore\ 
\boldsymbol{\tau}		&= \boldsymbol{\tau_0}	+\boldsymbol{\tau_g}\\
						&= \boldsymbol{\tau_0}	+
						   \boldsymbol{r} \times \boldsymbol{f_g}\\
\end{split}
\end{equation}
Then, Substitute the first line of Equation \ref{eq:gc_iga} into the above equation, we will obtain
\begin{equation}
\begin{split}
\boldsymbol{\tau}	&=  \boldsymbol{\tau_0}+ 
						\boldsymbol{p}	\times \left( \boldsymbol{f} - \boldsymbol{f_0} \right) 
\end{split}
\end{equation}
, which could be extended as
\begin{equation}
\begin{split}
\left\{\begin{matrix}
\tau _{x}	=	\tau _{0x} 	+ \left( f_z - f_{0z}\right) \cdot y - \left( f_y - f_{0y}\right) \cdot z\\
\tau _{y}	=	\tau _{0y}	+ \left( f_x - f_{0x}\right) \cdot z - \left( f_z - f_{0z}\right) \cdot x\\
\tau _{z}	=	\tau _{0z}	+ \left( f_y - f_{0y}\right) \cdot x - \left( f_x - f_{0x}\right) \cdot y& 
\end{matrix}\right.	\\
\end{split}
\end{equation}
and be overwritten as
\begin{equation}\label{eq:matrix_mfrk}
\begin{split}
\begin{bmatrix}
\tau _x\\
\tau _y\\
\tau _z
\end{bmatrix}
=
\begin{bmatrix}
0		&f_z	&-f_y	&1	&0	&0\\
-f_z	&0		&f_x	&0	&1	&0\\
f_y		&-f_x	&0		&0	&0	&1
\end{bmatrix}
\begin{bmatrix}
x\\
y\\
z\\
k_1\\
k_2\\
k_3
\end{bmatrix}\\
\end{split}
\end{equation}
\par
where
\begin{equation}\label{eq:k1k2k3}
\begin{split}
\left\{\begin{matrix}
k_1 = \tau _{0x} - \left( f_{0z} \cdot y + f_{0y} \cdot z \right)  &  \\
k_2 = \tau _{0y} - \left( f_{0x} \cdot z + f_{0z} \cdot x \right)  &\text{are all constant} \\
k_3 = \tau _{0z} - \left( f_{0y} \cdot x + f_{0x} \cdot y \right)  & 
\end{matrix}\right.\\
\end{split}
\end{equation}
With extracting $[x,y,z,k_1,k_2,k_3]$ in mind, we move the robot arm to $n\ (n\geq3)$ positions with different poses. By recording $n$ torque vectors and corresponding $n$ force vectors from F/T sensor, we can expand Equation \ref{eq:matrix_mfrk} as
\begin{equation}
\begin{split}
\begin{bmatrix}
\tau _x^1	\\
\tau _y^1	\\
\tau _z^1	\\
\tau _x^2	\\
\tau _y^2	\\
\tau _z^2	\\
\vdots\\
\tau _x^n	\\
\tau _y^n	\\
\tau _z^n	
\end{bmatrix}
=
\begin{bmatrix}
0			&f_z^1		&-f_y^1		&1	&0	&0\\
-f_z^1		&0			&f_x^1		&0	&1	&0\\
f_y^1		&-f_x^1		&0			&0	&0	&1\\
0			&f_z^2		&-f_y^2		&1	&0	&0\\
-f_z^2		&0			&f_x^2		&0	&1	&0\\
f_y^2		&-f_x^2		&0			&0	&0	&1\\
 			& 			&\vdots		& 	& 	& \\
0			&f_z^3		&-f_y^3		&1	&0	&0\\
-f_z^3		&0			&f_x^3		&0	&1	&0\\
f_y^3		&-f_x^3		&0			&0	&0	&1
\end{bmatrix}
\begin{bmatrix}
x\\
y\\
z\\
k_1\\
k_2\\
k_3
\end{bmatrix}\\
\end{split}
\end{equation}
\par
, which is defined as 
\begin{equation}
\begin{split}
\boldsymbol{m}_{\left(3n \times 1\right)} = \mathbf{F}_{\left(3n \times 6\right)} \cdot \boldsymbol{p}_{\left(6 \times 1\right)}
\end{split}
\end{equation}
As a consequence of the full column rank of $\mathbf{F}$, we can apply Moore-Penrose pseudoinverse. Then, 
\begin{equation*}
\begin{split}
\boldsymbol{p} 	&= \mathbf{F}^{\dagger} \cdot \boldsymbol{m}\\
				&= \left( \mathbf{F}^\top\mathbf{F}\right) ^{-1}\mathbf{F}^\top \cdot \boldsymbol{m}
\end{split}
\end{equation*}
\par
From now on, we have already known the centroid position of end effector in the sensor frame and values of the constants $k_1,k_2,k_3$.
\subsubsection{Gravity Compensation and Initial Value Reset}
\hspace*{5mm}Next, let us return to the first equation in Figure \ref{fig:gravity compensation}. We continue to use this formula and contemplate it from the perspective of coordinate transformation relation. Here we hypothesize that the end effector weighs $g$ kilograms relative to the frame\{0\} which is also the world frame. That means the gravity vector of the end effector $\boldsymbol{^0\!g}$ is $[0,0,-g]$ in the frame\{0\}. Then, we can derive it as following.
\begin{equation}
\begin{split}
\text{Initial value}	+ \text{Gravity Load} 		&= \text{Actual Value} \\
\Rightarrow
\boldsymbol{f_0} +\  ^\mathrm{S}_6\mathbf{R} \cdot ^6_0\!\mathbf{R} \cdot \boldsymbol{^0\!g} &= \boldsymbol{f}
\end{split}
\end{equation}
\par
which is relative to frame \{6\}.
Hence, we assume an installation angle $\theta$ including assembly error.
\begin{equation}
\begin{split}
 ^\mathrm{S}_6\mathbf{R}
=
\begin{bmatrix}
\cos(\theta)	&\sin(\theta)	&0 \\
-\sin(\theta)	&\cos(\theta)	&0 \\
0				&0				&1
\end{bmatrix}
\end{split}
\end{equation}
Besides, we can easily calculate $^6_0\mathbf{R}$ from Equation \ref{eq:translation matrix}.
\begin{equation}
\begin{split}
^6_0\mathbf{R} 	&=\ ^0_6\mathbf{R}^\top\\
				&\coloneqq
\begin{bmatrix}
r_{11}		&r_{12}		&r_{13} \\
r_{21}		&r_{22}		&r_{23} \\
r_{31}		&r_{32}		&r_{33}
\end{bmatrix}
\end{split}
\end{equation}
Therefore,
\begin{equation}
\begin{split}
\begin{bmatrix}
f_{0x}\\
f_{0y}\\
f_{0z}
\end{bmatrix}
+
\begin{bmatrix}
\cos(\theta)	&\sin(\theta)	&0 \\
-\sin(\theta)	&\cos(\theta)	&0 \\
0				&0				&1
\end{bmatrix}
\begin{bmatrix}
r_{11}		&r_{12}		&r_{13} \\
r_{21}		&r_{22}		&r_{23} \\
r_{31}		&r_{32}		&r_{33}
\end{bmatrix}
\begin{bmatrix}
0\\
0\\
-g
\end{bmatrix}
=
\begin{bmatrix}
f_x\\
f_y\\
f_z
\end{bmatrix}
\end{split}
\end{equation}
\par
Further, we rewrite it as 
\begin{equation}
\label{eq:matrix_rgf0f}
\begin{split}
\begin{bmatrix}
-r_{13}		&-r_{23} 	&0			&1		&0		&0\\
-r_{23}		&r_{13}		&0			&0		&1		&0\\
0			&0			&-r_{33}	&0		&0		&1
\end{bmatrix}
\begin{bmatrix}
g\cos(\theta)\\
g\sin(\theta)\\
g\\
f_{0x}\\
f_{0y}\\
f_{0z}
\end{bmatrix}
=
\begin{bmatrix}
f_x\\
f_y\\
f_z
\end{bmatrix}
\end{split}
\end{equation}
In the same way as Equation \ref{eq:matrix_mfrk}, we can extract $[g\cos(\theta),g\sin(\theta),g,f_{0x},f_{0y},f_{0z}]$ by the least square solution. Note that, we can simultaneously record these data when using the method in Equation \ref{eq:matrix_mfrk}. By recording $n(n\geq3)$ third columns of ration matrices and corresponding $n$ force vectors from F/T sensor, we can expand Equation \ref{eq:matrix_rgf0f} as
\begin{equation}
\begin{split}
\begin{bmatrix}
-r_{13}^1		&-r_{23} ^1		&0			&1		&0		&0\\
-r_{23}^1		&r_{13}^1		&0			&0		&1		&0\\
0				&0				&-r_{33}^1	&0		&0		&1\\
-r_{13}^2		&-r_{23}^2 		&0			&1		&0		&0\\
-r_{23}^2		&r_{13}^2		&0			&0		&1		&0\\
0				&0				&-r_{33}^2	&0		&0		&1\\
&&\vdots\\
-r_{13}^n		&-r_{23}^n 		&0			&1		&0		&0\\
-r_{23}^n		&r_{13}^n		&0			&0		&1		&0\\
0				&0				&-r_{33}^n	&0		&0		&1
\end{bmatrix}
\begin{bmatrix}
g\cos(\theta)\\
g\sin(\theta)\\
g\\
f_{0x}\\
f_{0y}\\
f_{0z}
\end{bmatrix}
=
\begin{bmatrix}
f^1_{0x}\\
f^1_{0y}\\
f^1_{0z}\\
f^2_{0x}\\
f^2_{0y}\\
f^2_{0z}\\
\vdots\\
f^n_{0x}\\
f^n_{0y}\\
f^n_{0z}
\end{bmatrix}
\end{split}
\end{equation}
\par
, which is defined as 
\begin{equation}
\begin{split}
\mathbf{M}_{\left(3n \times 6\right)} \cdot \boldsymbol{p}_{\left(6 \times 1\right)} = \boldsymbol{f}_{\left(3n \times 1\right)}
\end{split}
\end{equation}
As a consequence of the full column rank of $\mathbf{M}$, we can apply Moore-Penrose pseudoinverse. Then, 
\begin{equation*}
\begin{split}
\boldsymbol{p} 	&= \mathbf{M}^{\dagger} \cdot \boldsymbol{f}\\
				&= \left( \mathbf{M}^\top\mathbf{M}\right) ^{-1}\mathbf{M}^\top \cdot \boldsymbol{f}
\end{split}
\end{equation*}
Apparently, we can directly obtain $g,f_{0x},f_{0y},f_{0z}$. Afterwards, we substitute them into Equation \ref{eq:k1k2k3} to calculate 
\begin{equation}
\begin{split}
\left\{\begin{matrix}
\tau _{0x}	=	k_1	+ \left( f_{0z} \cdot y + f_{0y} \cdot z \right) \\
\tau _{0y} 	=	k_2	+ \left( f_{0x} \cdot z + f_{0z} \cdot x \right) \\
\tau _{0z} 	=	k_3 + \left( f_{0y} \cdot x + f_{0x} \cdot y \right)
\end{matrix}\right.\\
\end{split}
\end{equation}
Finally, we successfully obtain the weight of the end effector $g$, the initial value of the F/T sensor $[f_{0x},f_{0y},f_{0z},\tau_{0x},\tau_{0y},\tau_{0z}]$.
\par
As for installation angle $\theta$ including assembly error,  there is an further discussion. Undoubtedly, we can derive it as
\begin{equation}
\begin{split}
\theta = \cos^{-1}\left(\frac{g\cos(\theta)}{g}\right)\ \text{or} \ \sin^{-1}\left(\frac{g\sin(\theta)}{g}\right)\
\end{split}
\end{equation}
In theory, By either the arccos function or arcsin function, we can derive the same value. However, we estimate $[g\cos(\theta),g\sin(\theta),g,f_{0x},f_{0y},f_{0z}]$ by using the least square solution which produce a approximated answer rather than the correct answer. A subtle bias caused by the least square solution will be enlarged through the arc function. 
Thankfully, we originally designed an adapter to connect the robot arm and F/T sensor, and the installation angle is exactly zero degrees. Therefore, here we only need to concern about the subtle bias result from assembly errors.\\
Here we take the same bias and compare arccos to arcsin. 
\begin{table}[htbp]
\centering
\caption{Arc-function Comparison}
\label{tab:arc}
\begin{tabular}{c|c|c} 
\hline \hline
bias $n$ (rad)	&$\sin^{-1}(n)$	(degree)	&$\cos^{-1}(1-n)$ (degree)\\
\hline
0				&0							&0\\
0.001			&0.057						&2.56\\
0.01			&0.57						&8.11\\
0.1				&5.7						&25.84\\
\hline\hline
\end{tabular}
\end{table}
Obviously, if we separately give the same bias ( $0.1$ rad) into the arccos and arcsin function, the arccos function will enlarge the bias significantly larger than the arcsin function. Hence we should use
\begin{equation}
\begin{split}
\theta = \sin^{-1}\left(\frac{g\sin(\theta)}{g}\right)\
\end{split}
\end{equation}
Notably, as a result of the zero installation angle we initially designed from Table \ref{tab:arc}, we could infer using the arcsin function. In contrast, if the installation angle is not zero, the above assumption will be invalid.
\par
Ultimately, we have successfully dissected the actual value of the F/T sensor into the initial value of the F/T sensor and the gravity of the end effector. 
\subsubsection{External Force Evaluation}
Next, we continue to review the second equation in  Figure \ref{fig:gravity compensation}.
\begin{equation}
\begin{split}
\text{Initial value}	+ \text{Gravity Load} 		+ \text{External Force}		&= \text{Actual Value} \\
\Rightarrow \left\{\begin{matrix}
\boldsymbol{f_0}		+\boldsymbol{f_g}		+\boldsymbol{f_e}		&= \boldsymbol{f}\\ 
\boldsymbol{\tau_0}		+\boldsymbol{\tau_g}	+\boldsymbol{\tau_e}		&= \boldsymbol{\tau}	
\end{matrix}\right.	\\
\end{split}
\end{equation}
Therefore,
\begin{equation}
\begin{split}
\boldsymbol{f_e}		&= \boldsymbol{f} 		- \boldsymbol{f_0}		-\boldsymbol{f_g}\\
						&= \boldsymbol{f} 		- \boldsymbol{f_0}		- ^\mathrm{S}_6\mathbf{R} \cdot ^6_0\!\mathbf{R} \cdot \boldsymbol{^0\!g}\\
\boldsymbol{\tau_e}		&= \boldsymbol{\tau}	- \boldsymbol{\tau_0}		-\boldsymbol{\tau_g}\\
						&= \boldsymbol{\tau}	- \boldsymbol{\tau_0}		-\boldsymbol{r} \times \boldsymbol{f_g}
\end{split}
\end{equation}
\begin{equation*}
\begin{split}
\Rightarrow 
\left\{\begin{matrix}
f_{ex} = f_x - f_{0x} 	+ g\cos(\theta)r_{13} 	+ g\sin(\theta)r_{23}\\
f_{ey} = f_y - f_{0y} 	- g\sin(\theta)r_{13} 	+ g\cos(\theta)r_{23}\\
f_{ez} = f_z - f_{0z} 	+ gr_{33}\\
\tau_{ex} = f_x - \tau_{0x} - \left( g_z\cdot y - g_y\cdot z \right)\\
\tau_{ey} = f_y - \tau_{0y} - \left( g_x\cdot z - g_z\cdot x \right)\\
\tau_{ez} = f_z - \tau_{0z} - \left( g_y\cdot x - g_x\cdot y \right)
\end{matrix}\right.
\end{split}
\end{equation*}
\section{Dragging Mode}
\hspace*{6mm}In the hope that dentist could drag our system to an infected tooth by holding the end effector, we usher in admittance control based on F/T sensor.
\subsection{Admittance Control based on F/T sensor}
\label{sec:adm ctrl}
\subsubsection{Control Scheme}
\hspace*{6mm}Admittance control makes the robot move like a spring-mass-damper system. Forces and torques can be mapped into the movements such as position or velocity. Most importantly, admittance control enables a robot arm to cooperate with humans in a safe work environment. Since Meca500 is an industrial robot arm without admittance control, we subsequently combine the robot arm with the F/T sensor to adopt admittance control. With force and torque feedback, the F/T sensor makes Meca500 be like a collaborative robot arm. Therefore, we propose a control scheme depicted as Figure \ref{fig:adm ctrl}. It's worth noting that in this approach, the admittance control function is triggered by the end effector mounted on the F/T sensor instead of detecting each wrist torque of the robot arm.
\par
\begin{figure}[htbp]
\begin{center}
\includegraphics[width=1\linewidth]{Images/adm ctrl.png}
\caption{
Control scheme. $\boldsymbol{f_d}$ denotes the desired forces and torques vector. $\boldsymbol{f_s}$ denotes the real value detected by F/T sensor and is also a forces and torques vector. $\boldsymbol{\dot{x}}$ denotes [$\dot{x}, \dot{y}, \dot{z}, \dot{\theta _x}, \dot{\theta _y}, \dot{\theta _z}$]. $\mathbf{J_g}$ denotes the geometric Jacobian matrix. $\boldsymbol{\dot{q}}$ denotes [$\dot{\theta _1}, \dot{\theta _2}, \dot{\theta _3}, \dot{\theta _4}, \dot{\theta _5}, \dot{\theta _6}$]. $\boldsymbol{q}$ denotes $\left[\theta _1, \theta _2, \theta _3, \theta _4, \theta _5, \theta _6 \right] $.
}\label{fig:adm ctrl}
\end{center}
\end{figure}
\par
A standard equation of admittance control is represented as Equation \ref{eq:adm_mbk}. The values we obtain from the F/T sensor are $\left[f_x, f_y, f_z,\tau _x, \tau _y, \tau _z \right]$, whose forces $ \left[f_x, f_y, f_z\right]$ are related to the translations $ \left[x, y, z\right]$ and torques $ \left[\tau _x, \tau _y, \tau _z\right]$ are related to the axis angle $ \left[\theta _x,\theta _y,\theta _z\right]$. 
\begin{equation}
\label{eq:adm_mbk}
\begin{split}
\begin{bmatrix}
x \\
y \\
z \\
\theta _x \\
\theta _y \\
\theta _z 
\end{bmatrix}
=
\frac{1}{\mathbf{M}s^2+\mathbf{B}s+\mathbf{K}}
\begin{bmatrix}
f_x \\
f_y \\
f_z \\
\tau _x \\
\tau _y \\
\tau _z 
\end{bmatrix}
\end{split}
\end{equation}
\subsubsection{Selection of Admittance Control Model}
\hspace*{6mm}In our proposed approach, we omit parameter $\mathbf{K}$ which is relevant to spring stiffness, considering that it's not necessary to bounce such as a spring. Therefore, our system should behave like a mass-damper system as following.
\begin{equation}
\label{eq:adm_mb}
\begin{split}
\begin{bmatrix}
x \\
y \\
z \\
\theta _x \\
\theta _y \\
\theta _z 
\end{bmatrix}
=
\frac{1}{\mathbf{M}s^2+\mathbf{B}s}
\begin{bmatrix}
f_x \\
f_y \\
f_z \\
\tau _x \\
\tau _y \\
\tau _z 
\end{bmatrix}
\end{split}
\end{equation}
where $\mathbf{M},\mathbf{B},\mathbf{K}$ are diagonal positive definite matrices. Affections of these parameters will be discussed in section \ref{sec:affection}.
\subsubsection{Robot Command Decision}
\hspace*{6mm}After determining our admittance control model, we should select a correspondent command to move the robot. However, there are many commands of moving the robot arm, as shown in Table \ref{tab:commands}.
\begin{table}[htbp]
\centering
\caption{Moving commands in Meca500}
\label{tab:commands}
\begin{tabular}{c|c|l} 
\hline \hline
Types										&Commands			&Input parameters\\
\cline{1-3}
\multirow{2}{*}{Position}					&MoveJoints			&\noindent $\theta _1, \theta _2 ,\theta _3 ,\theta _4 ,\theta _5 , \theta _6 $\\\cline{2-3}
											&MovePose			&\noindent$x,y,z,\alpha ,\beta ,\gamma $\\
\cline{1-3}
\multirow{2}{*}{Velocity}					&MoveJointsVel		&$\dot{\theta}_1, \dot{\theta}_2, \dot{\theta}_3,\dot{\theta}_4, \dot{\theta}_5 , \dot{\theta}_6 $\\\cline{2-3}
											&MoveLinVelTRF		&$\dot{x},\dot{y},\dot{z},\dot{\theta _x},\dot{\theta _y},\dot{\theta _z}$\\
\hline \hline
\end{tabular}
\end{table}
\par
Considering that the singularity problem is an imperative subject in robotics, we intend to use MoveJoints or MoveJointsVel to set the axes' angles directly. Despite that it's easier to implement admittance control via other commands, the system would touch the singularity point at any time. It undoubtedly exposes patients to danger because of the uncertainty of the system. In our case, we could detect whether it is a singularity or not before commanding the robot. Therefore, position command - MoveJoints and velocity command - MoveJointsVel is our option. Why we choose the command MoveJoints is that Meca500 has a default time-out value to ensure its safety. Even though we could set this value from 0.001 to 2 seconds, Meca500 is still restricted to move with this value. For example, the value of time-out is set $0.1$ sec. While the robot receives a command, the robot move for $0.1$ sec then immediately stops. It's not easy to control via velocity command due to this default property.
\par
Finally, we designate MoveJointsVel $\left(\dot{\theta}_1, \dot{\theta}_2,\cdots , \dot{\theta}_6 \right)$ as our main command. Thanks to the property of Jacobian matrix shown in Equation \ref{eq:jg6}, we can transform $\boldsymbol{\dot{x}}$ into $\boldsymbol{\dot{q}}$. Then we multiply it an integrator $\frac{1}{\mathrm{S}}$
to obtain $\boldsymbol{q}$. Note that, As F/T sensor is mounted on the frame\{6\}, we should use Equation \ref{eq:jg6} rather than Equation \ref{eq:jg0}. 
\subsubsection{Implementation in Discrete Time}
\quad
To implement it in practice, we use the bilinear transformation, also known as Tustin method, to control it in discrete time.
\begin{equation}
\label{eq:z_domain eq}
\begin{split}
\frac{\boldsymbol{v}(s)}{\boldsymbol{f}(s)}
				&= \left( \mathbf{M}s+\mathbf{B}\right)^{-1}			\\
\Rightarrow 
\frac{\boldsymbol{v}(z)}{\boldsymbol{f}(z)}
				&= \frac{\mathrm{T}(1+z^{-1})}{(\mathrm{T}\cdot\mathbf{B}+2\mathbf{M})+(\mathrm{T}\mathbf{B}-2\mathbf{M})\cdot z^{-1}}
\end{split}
\end{equation}
\quad and,
\begin{equation}
\begin{split}
\frac{\boldsymbol{\theta}(s)}{\boldsymbol{w}(s)}
				&= \frac{\mathbf{K}}{s}				\\
\Rightarrow 
\frac{\boldsymbol{\theta}(z)}{\boldsymbol{w}(z)}
				&= \frac{\mathrm{T}\mathbf{K}}{2}	
					\frac{1+z^{-1}}{1-z^{-1}}
\end{split}
\end{equation}
Therefore,
\begin{equation}
\begin{split}
&\boldsymbol{v}[k+1] = \frac{\mathrm{T}}{\mathrm{T}\mathbf{B}+2\mathbf{M}}
							\cdot \left(	\boldsymbol{f}[k+1] + \boldsymbol{f}[k]		\right)
						 - \frac{\mathrm{T}\mathbf{B}-2\mathbf{M}}{\mathrm{T}\mathbf{B}+2\mathbf{M}}
						 	\cdot \boldsymbol{v}[k]												\\
&\boldsymbol{\theta}[k+1] = \boldsymbol{\theta}[k] + \frac{\mathrm{T}\mathbf{K}}{2}
												\left( \boldsymbol{w}[k+1] + \boldsymbol{w}[k] \right)
\\
&\text{where } \mathrm{T} \text{ is the sampling time.												}
\end{split}
\end{equation}
\section{Self-Alignment Mode}
\hspace*{6mm}This section aims to develop a framework for robot self-alignment regarding the position and orientation of the root canal, which is one of the main contributions of the thesis. This function assists endodontists in operating the cleaning procedure automatically and ensures a postoperative recovery for patients. The main idea is to amend the direction of insertion by itself to lower the contact resistance.
\par
We propose a complete robotic procedure. First and foremost, the prominent role of this procedure is the root canal reamer. We rotate the root canal reamer to clean a pulp, move it to do reciprocation and self-alignment. So, we care about the root canal, such as its position, orientation, rotation speed, and contact force. Nevertheless, the robot arm and the F/T sensor have their coordinates initially. They originally recognize its own coordinates rather than the root canal reamer's frame, so they do their work respectively. Therefore, we need to let them understand the tool frame. Thanks to section \ref{sec:ref_robot}, we have demonstrated how to change the reference frame of the robot arm. After that, we can identify the translation and rotation information of the root canal reamer and subsequently obtain the transformation matrix from the robot arm to the tool. As for the F/T sensor, we have explained how to do gravity compensation in section \ref{sec:grav compen}, we will interpret how to change its fiducial to the tooltip in section \ref{sec:rfc}. Besides, we will advance motion planning in accordance with the dentist's motion and standard surgical solution in section \ref{sec:motion planning} 
\subsection{Coordinate Transformation of F/T sensor}
\label{sec:rfc}
\hspace*{6mm}Similar to Coordinate transformation of F/T sensor in section \ref{sec:ref_robot}. The F/T sensor originally recognizes its coordinate rather than the root canal reamer's frame. Namely, the F/T sensor receives unmatched data when the tool contacts an obstacle due to the wrong reference frame. Hence, here we illustrate how to obtain the coordinate transformation of the F/T sensor to decouple the force and toque from frame\{S\} to the frame\{T\}. We analyze it from two perspectives - reference frame changing and measurement point changing.
\par
\begin{figure}[htbp]
\begin{center}
\includegraphics[width=0.7\linewidth]{Images/sensor_comp 1.png}
\caption{
Illustration of changing reference frame from $\{\mathrm{S}\} $ to $\{\mathrm{T}\}$. $\mathrm{f}$ and $\mathrm{m}$ respectively denote a force and a moment measured at point p.
}\label{fig:sensor_comp1}
\end{center}
\end{figure}
\par\noindent
In view of reference frame changing in Figure \ref{fig:sensor_comp1}, we can derive the following equation.
\begin{equation}
\begin{split}
\begin{bmatrix}
\boldsymbol{f}_\mathrm{p}\\ 
\boldsymbol{m}_\mathrm{p}
\end{bmatrix}
_{\{ \mathrm{T}\}}
=
\begin{bmatrix}
_\mathrm{S}^\mathrm{T}\mathbf{R} & \boldsymbol{0}\\ 
\boldsymbol{0} & _\mathrm{S}^\mathrm{T}\mathbf{R}
\end{bmatrix}
\begin{bmatrix}
\boldsymbol{f}_\mathrm{p}\\ 
\boldsymbol{m}_\mathrm{p}
\end{bmatrix}
_{\{ \mathrm{S}\}}
\end{split}
\end{equation}
\par
\begin{figure}[htbp]
\begin{center}
\includegraphics[width=0.7\linewidth]{Images/sensor_comp 2.png}
\caption{
Illustration of changing reference frame from $\{\mathrm{S}\} $ to $\{\mathrm{T}\}$. $\boldsymbol{f}$ and $\boldsymbol{m}$ respectively denote a force and a moment measured at point p.
}\label{fig:sensor_comp2}
\end{center}
\end{figure}
\par\noindent
On the other hand, in Figure \ref{fig:sensor_comp2} we describe a force and a torque observed in the same frame but at a different point.
\begin{equation}
\begin{split}
\boldsymbol{f}				&= \boldsymbol{f}_\mathrm{p} = \boldsymbol{f}_\mathrm{q}\\
\boldsymbol{m}_\mathrm{q} 	&= \boldsymbol{r}_\mathrm{q} \cdot \boldsymbol{f}\\
			 				&= \boldsymbol{r}_\mathrm{p} \cdot \boldsymbol{f}+\overrightarrow{\mathrm{q}\mathrm{p}} \times \boldsymbol{f}_\mathrm{p}\\
			 				&= \boldsymbol{m}_\mathrm{p} + \overrightarrow{\mathrm{q}\mathrm{p}}\times \boldsymbol{f}_\mathrm{p}
\end{split}
\end{equation}
Assume
\begin{equation*}
\begin{split}
\overrightarrow{\mathrm{q}\mathrm{p}}
=
\begin{bmatrix}
u_1\\
u_2\\
u_3
\end{bmatrix}
\end{split}
\end{equation*}
then
\begin{equation}
\begin{split}
\overrightarrow{\mathrm{q}\mathrm{p}}\times \boldsymbol{f}_\mathrm{p}
=
\begin{bmatrix}
0		&-u_3		&u_2		\\
u_3		&0			&-u_1		\\
-u_2	&u_1		&0		
\end{bmatrix}
\begin{bmatrix}
f_1\\
f_2\\
f_3
\end{bmatrix}
\end{split}
\end{equation}
As s result, we consider reference frame changing and measurement point changing, and then we can obtain the following essential equation.
\begin{equation}
\begin{split}
\begin{bmatrix}
\boldsymbol{f}_\mathrm{q}\\ 
\boldsymbol{m}_\mathrm{q}
\end{bmatrix}
_{\{ \mathrm{T}\}}
=
\begin{bmatrix}
\mathbf{I}_{3 \times 3} & \boldsymbol{0}\\ 
\begin{matrix}
0		&-u_3		&u_2		\\
u_3		&0			&-u_1		\\
-u_2	&u_1		&0		
\end{matrix} & \mathbf{I}_{3 \times 3}
\end{bmatrix}
\begin{bmatrix}
_\mathrm{S}^\mathrm{T}\mathbf{R} & \boldsymbol{0}\\ 
\boldsymbol{0} & _\mathrm{S}^\mathrm{T}\mathbf{R}
\end{bmatrix}
\begin{bmatrix}
\boldsymbol{f}_\mathrm{p}\\ 
\boldsymbol{m}_\mathrm{p}
\end{bmatrix}
_{\{ \mathrm{S}\}}
\end{split}
\end{equation}
\subsection{Motion Planning: Based on Admittance Control}
\label{sec:motion planning} 
Finally, we plan a series of motions to simulate the cleaning procedure of the root canal treatment. Endodontist uses their professional experience to determine how many force to apply and when to reverse the file to release the torque caused by the contact force. Because the root canal resembles a cone, the profundity of the file is larger, the resultant force composed of contact forces is more significant. Namely, the endodontist will apply more and more pressure in the insertion direction during the surgery. To simulate this series of motions, we could directly command the robot arm to move with the series of actions. However, commanding the robot arm is a dilemma because we have already use admittance control which also commands the robot arm to move. Therefore,  to solve the conflict, we continue to utilize admittance control and propose a method based on admittance control to simulate the series of motions shown in Fig \ref{fig:motion planning_flow chart}.
\begin{figure}[htbp]
\begin{center}
\includegraphics[width=0.5\linewidth]{Images/motion planning_flow chart.png}
\hspace*{6mm}\caption{
Flow chart of motion planning
}\label{fig:motion planning_flow chart}
\end{center}
\end{figure}
\begin{figure}[htbp]
\begin{center}
\includegraphics[width=0.5\linewidth]{Images/motion planning_motion.png}
\caption{
Flow chart of motion planning
}\label{fig:motion planning_motion}
\end{center}
\end{figure}
\section{Affections of Parameters Setting}
\label{sec:affection}
\hspace*{6mm}Because our system is similar to a mass- damper system, we focus on the performance of the velocity $\boldsymbol{\dot{x}}$ whereby the system could be considered as
\begin{figure}[htbp]
\begin{center}
\includegraphics[width=0.6\linewidth]{Images/mass spring.png}
\hspace*{6mm}\end{center}
\caption{
Control scheme. $\mathbf{M},\mathbf{B},\mathbf{K}$ are all diagonal positive definite matrices, whose diagonal indexes are respectively related to $x,y,z,\theta_x,\theta_y,\theta_z$. $\mathbf{M},\mathbf{B}$ are related to the inertial and damping respectively, and $\mathbf{K}$ is a proportional gain. 
}\label{fig:mass spring}
\end{figure}
\par\noindent
It is a first-order control system and its step response is 
\begin{equation}
\begin{split}
\mathcal{L}^{-1} \left[ \frac{1}{\mathrm{S}} \cdot \frac{\mathbf{K}}{\mathbf{M}\mathrm{S}+\mathbf{B}} \right] 
&= \mathcal{L}^{-1} \left[ \frac{\mathbf{K}}{\mathbf{B}} \left( \frac{1}{\mathrm{S}} - \frac{1}{\mathrm{S}+\frac{\mathbf{B}}{\mathbf{M}}} \right) \right] \\
&= \frac{\mathbf{K}}{\mathbf{B}} \left(1 - e^{- \frac{\mathbf{B}}{\mathbf{M}} t}  \right)
\end{split}
\end{equation}
From the above derivation, we can know 
\begin{equation}
\begin{split}
\text{time constant } \boldsymbol{\tau} &= \frac{\mathbf{M}}{\mathbf{B}}\\
\text{gain} &= \frac{\mathbf{K}}{\mathbf{B}}
\end{split}
\end{equation}
Hence,the transient response can be derived as 
\begin{equation}
\begin{split}
&\text{Rise time} = 2.3\boldsymbol{\tau} = \frac{2.3 \mathbf{M}}{\mathbf{B}}\\
&\text{Settling time} = 4\boldsymbol{\tau} =  \frac{4 \mathbf{M}}{\mathbf{B}}\\
\end{split}
\end{equation}
\par
In theory, when $\boldsymbol{\tau}$ is larger, the pole is farther from the origin in $\mathrm{S}$ domain, the system is more stable. Besides, when $\boldsymbol{\tau}$ is larger, the response speed is faster. Furthermore, when the gain $\frac{\mathbf{K}}{\mathbf{B}}$ is larger, the robot arm moves more. Therefore, we coarse-tune parameter $\mathbf{K}$ to adjust the whole gain of the system and determine the mode of the system is "Dragging Mode" or "Self-Alignment Mode". Last but not least, we fine-tune diagonal parameters of $\mathbf{B}$ separately because the inertial and spring properties of each axis are discrepant. 
\par
In practice, we implement it in discrete time. From Equation \ref{eq:z_domain eq}, we can further derive the final value in discrete time shown as 
\begin{equation}
\begin{split}
\text{gain} = \lim_{z \to 1} (z-1)\cdot \frac{\mathrm{T}\mathbf{K}(z+1)}{(\mathrm{T}\mathbf{B}+2\mathbf{M})z+\mathrm{T}\mathbf{B}-2\mathbf{M}}\cdot \frac{1}{\frac{2}{\mathrm{T}}\frac{z-1}{z+1}} = \frac{\mathbf{K}\mathrm{T}}{\mathbf{B}}
\end{split}
\end{equation}
Compared to the gain in continuous time, the sampling time in discrete time also effect the gain. Because the sampling time we set is $0.002$s, we set $(m_i : b_i = 1 : 500)$  to make $(\tau _i = \frac{m_i}{b_i} = 0.002)$, then the system will rise up to $98\%$ after $0.004$ second. Further, we can alter $\mathbf{K}$ to easily regulate the gain. In Table \ref{tab: para_adm}, we show detailed parameters with "Dragging Mode" and "Self-Alignment Mode".
\begin{table}[htbp]
\centering
\caption{Parameters setting of Admittance control.}
\label{tab: para_adm}
\begin{tabular}{ccc} 
\hline \hline
Parameter	&Dragging Mode		&Self-Alignment Mode	\\
\hline
k			&$200000$			&$100000$				\\
$b_1$		&$1.2$				&$0.7$					\\
$b_2$		&$0.6$				&$0.7$					\\
$b_3$		&$1.1$				&$0.7$					\\
$b_4$		&$1.7$				&$2$					\\
$b_5$		&$4$				&$2$					\\
$b_6$		&$1.8$				&$2$					\\
%$m_1$		&$0.004$			&$0.012$				\\
%$m_2$		&$0.004$			&$0.006$				\\
%$m_3$		&$0.004$			&$0.011$				\\
%$m_4$		&$0.008$			&$0.017$				\\
%$m_5$		&$0.008$			&$0.04$					\\
%$m_6$		&$0.008$			&$0.018$		
\hline	
&$\mathbf{K} = \text{diag}(k,k,k,k,k,k),$ &$\mathbf{B} = \text{diag}(b_1,b_2,b_3,b_4,b_5,b_6)$\\
\hline\hline	
\end{tabular}
\end{table}
\par
In "Dragging Mode", we hold the handpiece to move as shown in Fig. Due to different level arms, the forces we apply in each axis are all different. Therefore, to provide users a good experience, we set each parameter of $\mathbf{B}$ of "Dragging mode"  with different values. With these different values, the users can comfortably move the robot arm to the desired position and orientation. On the other hand, in "Self-Alignment Mode", an external force is applied at the tool, and we have already transformed the reference frame of the F/T sensor. We set the first three variables relative to the linear velocity as the same values. Besides, We set the last three variables relative to the angular velocity as the other same values.