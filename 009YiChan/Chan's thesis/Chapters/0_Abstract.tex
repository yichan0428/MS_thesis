%\addcontentsline{toc}{chapter}{Abstract} 
%\section*{\centering Abstract}
\chapter{Abstract}
\label{ch:abstract}
\vspace{3mm}
\hspace*{6mm}Advancements in robot-assisted surgery invigorate the application of robotic technologies in dentistry. This thesis aims to develop a robotic system in assisting endodontic treatment procedures. Considering the workspace and the extreme precision in endodontic treatment, a six degrees of freedom robotic manipulator -- DentiBot -- forms the basis of the developed system. Due to the lack of visual feedback in endodontic treatment, the DentiBot is integrated with a force and torque sensor, whose measurements guide the robot's motion during surgical procedures. The most critical two factors causing the failures of root canal treatment, namely incomplete root preparation and instrument fracture, are addressed. First, force-guided alignment based on admittance control techniques is proposed to adjust the surgical path and compensate for the patient's movement in real-time. Second, file federate control is proposed to protect endodontic files from fracturing, particularly when the file gets stuck in the root canal. These two functions are combined to achieve high performance in the root preparation procedure. Experimental results have demonstrated the performance of force-guided alignment and file federate control. The feasibility of robot-assisted endodontic treatment is verified by the pre-clinical evaluation performed on acrylic root phantoms.
\vspace{10mm}
\par\noindent
\textit{Keywords: Endodontic treatment, Robot-assisted system, Admittance control, Force-guided alignment, File feedrate control, Instrument fracture}




