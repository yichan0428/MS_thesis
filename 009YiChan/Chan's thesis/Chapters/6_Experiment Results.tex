\chapter{Preliminary Experiment Result}
\section{Experimental Setup}
Use present tense.\par\noindent
(Communication protocol – EtherCAT, RTOS – NI target)						
\par\noindent
For 6.2 experiment: (Stewart-Platform + PhaseSpace + markers)				
\par\noindent
For 6.3 6.4 experiments: (Acrylic root canal model + truth tooth)
\section{Admittance Control}
In order to prove the validation of admittance control, we set up this experiment. Here we built a Stewart platform, which had six degree of freedom and provided a small movement. We used Stewart platform to simulate a motion of patient. Basically, when the patient move to a position, our system should move to the same position. Therefore, we planned to observe the target's and the handpiece's position to validate that our system can track the patient. Besides, we use PhaseSpace to obtain their motion in real time. PhaseSpace is a motion capture device whose resolution is around $1$ mm. Before starting this experiment, we should fix a relative position between the file and the acrylic tooth. We made the file rotate and get stuck in the root canal of acrylic model. Then we used "Doctor dragging" mode to install the acrylic model on the Stewart platform. Therefore, we can guarantee that the relative position between the file and the acrylic tooth.
\par
We moved the Stewart platform in horizontal and vertical direction separately. The motion planning in horizontal direction is a square which is a $10\times 10$ mm and in vertical direction is a linear motion from $0$ to $40$ mm.
\section{Automatically Direction Changing}
Validation of Self-alignment Mode
\par\noindent
(Metrics: time, completeness and file breakage)								
\par\noindent
(Completeness definition: comparison of pixel area before and after experiment via image)
\section{Repetitive Experiment}
validation of repetitive experiment
\par\noindent
(Metrics: file breakage, compare with and without reverse)