\chapter{Conclusions and Future work}
\label{chapter7}
\hspace*{6mm}An endodontic treatment is a challenging surgery for dentists due to complex conditions of teeth. Therefore, building a robot-assisted system for the endodontic robot requires comprehensive consideration on whole surgery procedures. Review the section \ref{sec:contributions}, we advocated the four parts of the project prospect. We highlight two main parts of the prospect and have proved the feasibility of our proposed approaches. The whole work of the thesis is concluded in this chapter.
\section{Conclusions}
\hspace*{6mm}Nowadays, despite that there are more and more dental robots springing up, there are still few teams specific to endodontic treatment. In view of this, our team has built the robot-assisted system -- DentiBot -- composed of a 6-DoF robot arm, a 1-DoF modified handpiece, and a 6-DoF F/T sensor. System integration solutions between the above devices are reviewed. DentiBot can assist dentists in performing endodontic treatment as a consequence of the following functions. 
\par
Admittance control based on F/T sensor is implemented to enable dentists to move DentiBot above an infected tooth. Also, a framework based on admittance control for real-time force-guided alignment is presented to adjust surgical path and compensate patient moving. "Dragging Mode" and "Self-Alignment Mode" are separately implemented for the above functions.
\par
Moreover, instrument fracture is a severe iatrogenic error for dentists and might lead to a medical dispute. With the torque monitoring system, inverse rotation control and file feedrate control were applied to reduce the possibility of instrument fracture. 
\par
Experiments proved the feasibility in technical and clinical perspectives. Experiment 1 demonstrated the performance of force-guided alignment and file feedrate control. Experiment 2 is a pre-clinical evaluation and verify the feasibility of robot-assisted endodontoic treatment. Undoubtedly, DentiBot can help dentists perform better clinical results.
\section{Discussion and Future Work}
\hspace*{6mm}The thesis is the pioneer of the endodontic project. Despite that the thesis develops DentiBot and presents the above functions, there are some remained works and rooms for improvement on the modified hardware and functions. The modified handpiece made by 3-D prints is not durable for a time-consuming endodontic treatment. It is necessary to do machining for more stable results. 
\par 
In the thesis, we hypothesize that a dentist moves DentiBot above the root, then DentiBot does the "Cleaning" procedure. However, sometimes there is not only one root in the tooth. For instance, there are three to four roots in a molar. Therefore, DentiBot should have the ability to search all root canals after moved above the infected tooth. 
\par
On top of that, we proposed the alignment method while DentiBot does drilling. It not only aligns with the root path but also with patient moving. However, this patient tracking belongs to a minor range movement. Therefore, DentiBot is expected to be applied to patient tracking with large movement via string potentiometers in the future.			